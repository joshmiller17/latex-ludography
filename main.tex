\documentclass[sigconf]{acmart}
\usepackage[utf8]{inputenc}

% Copy the code below into the header of your document
% --- LUDOGRAPHY ---
\usepackage[resetlabels]{multibib}
\newcites{game}{Ludography}
\newcommand{\citegameprefix}{G}
\usepackage{xparse}
\let\origcitegame\citegame
\RenewDocumentCommand{\citegame}{O{} O{} m}{%
  \renewcommand{\citenumfont}[1]{\citegameprefix##1}%  
  \origcitegame[#1][#2]{#3}%
  \renewcommand{\citenumfont}[1]{##1}%
}
\newcommand{\citeg}[1]{\citegame{#1}} % shortcut
% ------------------


\title{Ludography Example}
\author{Josh Aaron Miller, Kutub Gandhi, and Seth Cooper}
\date{January 2023}

\begin{document}

\maketitle

\section{Introduction}

This \LaTeX{} source is an example of a ludography, featuring games like \emph{BioShock} (\citeyear{bioshock}) \citegame{bioshock}, \citeauthor{specops}'s  \emph{Spec Ops: The Line} \citegame{specops}, and \emph{Minecraft} \citegame{minecraft}. Ludographies are important sections of games research, such as in \cite{gandhi_philosophy_2022} and \cite{poretski2022press}.


% Copy the following section as a replacement for your bibliography code. Note that the "\bibliography{ref}" line should be changed depending on the name of your non-ludography .bib file
% --- LUDOGRAPHY ---

% set formatting for non-ludography references
\renewcommand{\bibnumfmt}[1]{[#1]}%  
\bibliographystyle{ACM-Reference-Format}
% change the following line depending on the name of your references file. For example, the following line is set up for a file called 'ref.bib'
\bibliography{ref}
% Note: this may produce warnings that game refs did not have a database entry

% set formatting for ludography
\renewcommand{\bibnumfmt}[1]{[\citegameprefix#1]}%  
\bibliographystylegame{ACM-Reference-Format}
\bibliographygame{games}

% -----------------
\end{document}
